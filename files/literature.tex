\section{Knowledge Management Foundations for Managing Quality Requirements}

In this paper, we consider requirements engineering as a knowledge management problem. 
Knowledge management addresses the acquisition of knowledge, transforming it from tacit or implicit into explicit knowledge and back again, storing, disseminating, and evaluating it systematically, and applying it in new situations \cite{Schneider2009}.
We see requirements as a special type of knowledge, which needs to be managed in an organization.
Doing requirements engineering then relates to organizational learning, which is an approach that stimulates learning of individuals, organization-wide collection of knowledge, and cultivation of infrastructure for knowledge exchange \cite{Schneider2009}.


%\eric{Classic KM, RBS, Cockburn...}

%\begin{itemize}
%\item Usabilty/security/safety: qualities differ, also their JIT behavior.
%\item Doors as a requirements database, a knowledge base of an LSO
%\end{itemize}

Nonaka's and Takeuchi's theory of knowledge creation relates to tacit and explicit knowledge \cite{Nonaka1995}\footnote{note that their notion of tacit knowledge differs from how this term is usually used in RE research: for them, tacit knowledge is not explicitly documented but can be shared in face-to-face communication.}.  
%
%We consider requirements engineering a crucial, knowledge generating activity.
Knowledge is created by converting it from a knowledge source (either tacit or explicit) to a new knowledge store (also either tacit or explicit). 
This view relates very well to requirements management, especially for quality requirements. 
For example, conversion of tacit knowledge to tacit knowledge (\emph{socialization} in Fig. \ref{fig:nonaka1995}), corresponds to an agile way of managing requirements, that de-emphasizes documentation and instead relies on face-to-face communication and just-in-time clarification of requirements. 
A long-term perspective on requirements would require explicit knowledge representations.
\emph{Combination}, for example, corresponds to tracing information derived from relating different artifacts of system engineering to each other. 
%Thus, we are especially interested in Nonaka's and Takeuchi's works on conversion between tacit and explit knowledge representations, as depicted in Fig. \ref{fig:nonaka1995}.

Earl has developed a framework to classify studies on knowledge management according to different research directions, which he calls schools \cite{Earl2001}.
The \emph{technocratic} school focuses on systems, maps and engineering of knowledge and resonates with a traditional approach to requirements engineering with a central requirements database (or specification) as knowledge base. 
In contrast, the \emph{economic} school focuses on commercial value of knowledge and the \emph{behavioral} school considers organizational, spatial, and strategic aspects.
We note that these latter schools resonate with values of agile RE.
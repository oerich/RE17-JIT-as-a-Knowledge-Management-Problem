\section{Introduction}

%\eric{4pages! title, abstract, intro to finish on page 1}

% \eric{Abstract, Intro: Basic assumption JIT+long-term; this is a KM problem}


%\eric{Context: More and more system development companies are turning towards agile...}
In this position paper, we argue based on our experience in several projects that while just-in-time management of quality requirements is important in agile development, it must be complemented by an initiative to manage long-term aspects of quality requirements and to build and use knowledge about the system under construction.


%Agile approaches have significantly changed the way software is developed \cite{Meyer2014}.
%Based on reports of successes with small teams \cite{Meyer2014,kahkonen2004agile,beck2000extreme,paasivaara2016challenges}, these approaches are more and more applied in large scale \cite{dikert2016challenges,lagerberg13,salo2008agile} and in system development \cite{eklund14,berger15,lagerberg13}, an environment that is characterized by long lead times \cite{berger15} and stable, sequential engineering practices \cite{pernstal12}.

%\eric{Quality requirements have JIT and long-term aspects and we cannot support one without knowing about the other. A holistic knowledge management approach is needed.}

Agile software development lacks systematic approaches for managing quality requirements \cite{inayat2015systematic} and needs further research, even though a variety of promising practices for managing quality requirements are known \cite{Alsaqaf2017}. 
Among those, some relate to just-in-time aspects of quality requirements engineering: relying on face-to-face communication and iterative emergence of requirements \cite{Alsaqaf2017}. 
Others imply a long-term perspective on quality requirements, as for example product grooming, continuous integration, and test-driven development \cite{Alsaqaf2017}.
Yet, existing challenges of managing quality requirements in agile development show an inability to synchronize just-in-time RE activities and long-term perspectives on architecture and system verification. 
This is concerning: on the one hand, we need just-in-time analysis of quality requirements to operationalize them for functionality currently under development.
On the other hand, quality requirements can be considered long-term business drivers that allow diversification from competitors \cite{BerntssonSvensson2015}.
A%s Cockburn argues, a
n agile team not only needs to care about the current %game (i.e. the current
release or project%, the current project)
, but also about the next %game
\cite{Cockburn2009}.

\textbf{Proposition 1} (JIT vs. Long-Term): \emph{We propose that JIT management of quality requirements must be complemented by an initiative to manage long-term aspects of quality requirements.}
%
% \eric{Agile System Development needs a user value and a system understanding perspective. This is based on \cite{Kasauli2017a}.}
%
In agile requirements engineering, a lot of emphasis is on understanding and communicating customer and end-user value \cite{Alahyari2016}.
This is important, since agile approaches rely on self-organized teams with some autonomy \cite{Meyer2014} and these teams need to understand what provides customer value, before they can make decisions \cite{Kasauli2017}.
Pre-agile approaches to requirements engineering emphasize the importance of distinguishing between user requirements and system requirements \cite{Sommerville2006} and this importance has been confirmed for requirements engineering in large-scale agile system development \cite{Kasauli2017a}. Incremental agile development and continuous delivery do not only require an excellent understanding of customer value, but also effective knowledge about how and why the current system was built. 

Consequently, many challenges in agile management of quality requirements relate to a lack of consideration of this system perspective in agile development, e.g. %\emph
{focusing on delivering functionality at the cost of architecture flexibility} as well as %\emph
{ignoring predictable architecture requirements} \cite{Alsaqaf2017}. 

\textbf{Proposition 2} (Customer value vs. System knowledge): \emph{ We claim that both JIT and long-term management of quality requirements must consider both a user (or: market) value perspective and a system requirements perspective.}
%
We argue that most qualities cannot be significantly improved just-in-time. 
While for example a single change can destroy security or safety of a system, the only way to create a system that has these properties is to grow it around a strong notion of the most important qualities. 
The knowledge how this has been done must be conserved and must be made accessible for JIT quality requirements activities.

In this position paper, we revisit well established knowledge management literature with our two key propositions in mind.
 

%In the following Section, we discuss foundations of knowledge management and how they relate to requirements engineering. In Section 3, we discuss the aspects and constraints of a suitable knowledge management framework for quality requirements in agile. 
%By this, we hope to enable more research on this important topic and discuss implications for future research in Section 3.

%\begin{itemize}
%    \item NEW: There exist different schools of knowledge management and we can conclude that there are alternatives to a technocratical, database centric approach. Behavioral and business perspectives are important in agile. (argue with EKSE book,  p60)
%  \item NEW: Business value in agile methods does very well fit into this view.
%
% \end{itemize}